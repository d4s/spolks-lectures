% СПО ЛКС  
% Пынькин Д.А. (с) 2013
\documentclass[ignorenonframetext, hyperref={pdftex, unicode}]{beamer}

\usepackage[russian]{babel}
\usepackage[utf8]{inputenc}
\usepackage[T1]{fontenc}

\usepackage{beamerthemesplit}

%\usetheme{Pittsburgh}
%\usecolortheme{dolphin}

\usetheme{Copenhagen}
\usecolortheme{wolverine}


\usepackage{ulem}
%\usepackage{html}

\usepackage{verbatim}

\usepackage{ucs}
\usepackage{listings}
\lstloadlanguages{C, make, bash}

\lstset{escapechar=`,
	extendedchars=false,
	language=C, 
	tabsize=2, 
	columns=fullflexible, 
%	basicstyle=\scriptsize,
	keywordstyle=\color{blue}, 
	commentstyle=\itshape\color{brown},
%	identifierstyle=\ttfamily, 
	stringstyle=\mdseries\color{green}, 
	showstringspaces=false, 
	numbers=left, 
	numberstyle=\tiny, 
	breaklines=true, 
	inputencoding=utf8x,
	keepspaces=true,
	morekeywords={u\_short, u\_char, u\_long, in\_addr}
	}

\definecolor{darkgreen}{cmyk}{0.7, 0, 1, 0.5}

\lstdefinelanguage{diff}
{
    morekeywords={+, -},
    sensitive=false,
    morecomment=[l]{//},
    morecomment=[s]{/*}{*/},
    morecomment=[l][\color{darkgreen}]{+},
    morecomment=[l][\color{red}]{-},
    morestring=[b]",
}

\title[СПОЛКС (http://goo.gl/32cTB)]{Системное программное обеспечение\\локальных компьютерных сетей}
\author{Денис Пынькин}
\date{2012 -- 2013}
%\institution[БГУИР]{Белорусский государственный университет информатики и радиоэлектроники}
%\logo{\includegraphics[width=1cm]{logo-kafEVM.png}}


\subtitle{Multicast}

\begin{document}

\mode<all>{\input{titlepage}}

%
% Далее начинается сама презентация
%

\begin{frame}{Multicast}

	\begin{block}{Широковещательный режим IP}
		\begin{itemize}
			\item one-to-many
			\item many-to-many
			\item realtime
			\item receiver driven
		\end{itemize}
	\end{block}

	\center\includegraphics[height=0.4\textheight]{800px-Multicast.png}

	Dave Clark: ``You put packets in at one end,  and the network conspires to deliver them to anyone who asks.``

\end{frame}

\begin{frame}{Multicast}

	Дейтаграмму многоадресной передачи должны получать только заинтересованные в ней интерфейсы. 
	
	Точнее приложения, которые используют эти интерфейсы и желают принять участие в сеансе многоадресной передачи данных.

\end{frame}

\begin{frame}{Маршрутизация}

	Маршрутизаторы общаются между собой при помощи какого-либо протокола маршрутизации многоадресной передачи
	(например DVMRP).


	\center\includegraphics[height=0.4\textheight]{IGMP_basic_architecture.png}
	
	Когда узел присоединяется к группе,  он отправляет всем маршрутизаторам в своей сети 
	сообщение IGMP (IPv4) или MLD (IPv6).

\end{frame}


\begin{frame}{IPv4 multicast}

	Используются адреса класса D:\\
	{\bf 224.0.0.0 -- 239.255.255.255}
	\center\includegraphics[height=0.4\textheight]{multicastgroups.png}

	\bigskip

	Младшие 28 бит -- идентификатор группы многоадресной передачи (multicast group ID)

\end{frame}

\begin{frame}{Специальные адреса IPv4}

	\begin{block}{224.0.0.1}

		группа всех узлов (all-hosts group)
	\end{block}

К этой группе должны присоединиться все узлы в сети,  имеющие возможность многоадресной передачи.
\end{frame}

\begin{frame}{Специальные адреса IPv4}

	\begin{block}{224.0.0.2}
		группа всех маршрутизаторов (all-routers group)
	\end{block}

	К этой группе должны присоединиться все маршрутизаторы сети,  поддерживающие многоадресную передачу.
	
\end{frame}

\begin{frame}{Специальные адреса IPv4}

	\begin{block}{224.0.0.0--224.0.0.255}
		Локальные адреса на канальном уровне (link local)
	\end{block}

	Предназначены для низкоуровневого определения топологии и служебных протоколов.
	\bigskip

	Дейтаграммы с такими адресами никогда не передаются маршрутизатором дальше.

\end{frame}


\begin{frame}{Ограничение области действия}

	Исторически сложилось,  что поле TTL в IPv4 выполняло роль поля области действия многоадресной передачи:

	\begin{itemize}
		\item 0: локальный в пределах узла
		\item 1: локальный в пределах сети
		\item \textless 32: локальный в пределах сайта
		\item \textless 64: локальный в пределах региона
		\item \textless 128: локальный в пределах континента
		\item \textless 255: глобальный
	\end{itemize}

\end{frame}

\begin{frame}{Ограничение области действия}

	Более предпочтительно использовать административное управление областями действия

	\begin{block}{239.0.0.0--239.255.255.255}
		пространство адресов с административным ограничением областью действия
	\end{block}

	\begin{itemize}
		\item 239.255.0.0--239.255.255.255: локальная область действия (site-local)
		\item 239.192.0.0--239.195.255.255: локальная область действия в пределах организации
	\end{itemize}

\end{frame}

\begin{frame}{Multicast IPv6 Address}

	\center\includegraphics[width=1\textwidth]{multicast_addr_ipv6.png}



\end{frame}

\begin{frame}{Специальные адреса IPv6}
	\begin{block}{FF02::1}
		группа всех узлов (all-nodes group)
	\end{block}

	К этой группе должны присоединиться все узлы в сети,  имеющие возможность многоадресной передачи
	(включая маршрутизаторы,  принтеры,  тостеры и холодильники,  если они имеют IPv6 адрес).

	\bigskip

	В отличие от IPv4 присоединение является обязательным!
\end{frame}

\begin{frame}{Специальные адреса IPv6}
	\begin{block}{FF02::2}
		группа всех маршрутизаторов (all-routers group)
	\end{block}

	К этой группе должны присоединиться все маршрутизаторы сети,  поддерживающие многоадресную передачу.

	Аналогичен адресу 224.0.0.2.

\end{frame}


\begin{frame}{Область действия: scope}
	\begin{itemize}
			\item 1: Node-local – локальная в пределах узла
			\item 12: Link-local – локальная в пределах физической сети
			\item 14: Admin-local – локальная в пределах области администрирования
			\item 15: Site-local – локальная в пределах сайта
			\item 18: Organization-local – локальная в пределах организации
			\item 114: глобальная
	\end{itemize}
	
	4, 5, 8 определяются администратором

\end{frame}


\begin{frame}{Сеансы многоадресной передачи}

	Сочетание адреса многоадресной передачи и порта транспортного уровня часто называют сеансом (session).
	\bigskip

	 Пример:\\
	 потоковое мультимедиа делится на 2 сеанса -- аудио и видео. 
	 При этом иногда выгодно использовать не только разные порты,  но и разные группы.
\end{frame}


\begin{frame}{Multicast и канальный уровень}

	Современные сетевые карты умеют принимать кадры с установленным битом режима многоадресной передачи.
	\bigskip
	Для беспроводных сетей 802.11,  передача кадра может задержаться в зависимости от
	режима сохранения энергии (power-save mode).

\end{frame}



\begin{frame}{Multicast: Ethernet \& IPv4}
	
	\center\includegraphics[width=1\textwidth]{multicast_ethernet.png}

	Используется универсально управляемый адрес Ethernet.

\end{frame}


\begin{frame}{Multicast: Ethernet \& IPv6}

	\center\includegraphics[width=1\textwidth]{multicast_ethernet_ipv6.png}

	Используется локально управляемый групповой адрес Ethernet.
\end{frame}


\begin{frame}{Multicast: программирование}
	Управление режимом,  присоединение к группам,  отсоединение и др. 
	осуществляется через вызовы {\bf getsockopt} и {\bf setsockopt}.
\end{frame}


\begin{frame}[fragile]{Multicast: структуры}
	\begin{block}{ip\_mreq}
		\begin{lstlisting}
struct ip_mreq{
	// адрес многоадресной передачи
	struct in_addr imr_multiaddr;
	// адрес интерфейса
	struct in_addr imr_interface;
}
		\end{lstlisting}
	\end{block}
	\begin{block}{ipv6\_mreq}
		\begin{lstlisting}
struct ipv6_mreq{
	// адрес многоадресной передачи
	struct in6_addr imr_multiaddr;
	// индекс интерфейса
	unsigned int ipv6mr_interface;
}
		\end{lstlisting}
	\end{block}
	
\end{frame}

\begin{frame}[fragile]{Multicast: структуры}
	\begin{block}{group\_req}
		\begin{lstlisting}
struct group_req{
	// индекс интерфейса
	unsigned int gr_interface;
	// адрес многоадресной передачи ipv4 или ipv6
	struct sockaddr_storage gr_group;
}
		\end{lstlisting}
	\end{block}
	
\end{frame}


\begin{frame}{Multicast}

\end{frame}

\begin{frame}{Multicast}

\end{frame}

\begin{frame}{Multicast}

\end{frame}

\begin{frame}{Multicast}

\end{frame}

\begin{frame}{Multicast}

\end{frame}




\mode<all>{\input{lastpage}}

\end{document}
%Конец файла
